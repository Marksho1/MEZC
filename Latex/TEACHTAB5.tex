\documentclass[10pt]{article}
\usepackage{array}
\usepackage{longtable}
\usepackage{fullpage}
\usepackage{dcolumn}
\usepackage{times}
\usepackage[flushleft]{threeparttable}
\usepackage{tabularx}
\usepackage{booktabs}
\usepackage[12hr]{datetime}
\usepackage{longtable}
\usepackage[DIV=16]{typearea}
\usepackage{scrextend,booktabs}
\usepackage{tabulary}
\usepackage{array}
\usepackage{multirow}
\usepackage[font=small]{caption}

\begin{document}
	
\begin{table}[H]
	\footnotesize
	\def\arraystretch{0.9}
	\centering
	\caption{Summary statistics - cognitive activation in mathematics}
\begin{tabulary}{1.0\textwidth}{L L C C C C}
	\hline\hline \\
	\multicolumn{2}{c}{}
	& \multicolumn{2}{c}{Dev7 countries}
	& \multicolumn{2}{c}{Vietnam}	\\
	\hline & & & & & & 
	Variable & Description & MS & Valid N &  MS & Valid N \\
	\hline \\

COGACT \textit{(r)} & Cognitive activation in & 0.2998 & 26217 & -0.3278 & 3249 \\ 
& mathematics lessons & (0.975) &  & (0.6647) &  \\ 
				
\hline \\
\multicolumn{6}{l}{Notes: The variables relate to the questionnaires administered to principals (schools) and}\\    
\multicolumn{6}{l}{students in the rotated booklet. For a more detailed description of variables, please see}\\
\multicolumn{6}{l}{Table xx. Items marked with \textit{(r)} are taken from the rotated student questionnaire. The}\\
\multicolumn{6}{l}{variable means of Dev7 and Vietnam are statistically different at the 5\% significance level.}\\

\end{tabulary}
\end{table}
	
	
\end{document}

