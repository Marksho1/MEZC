\documentclass[10pt]{article}
\usepackage{array}
\usepackage{longtable}
\usepackage{fullpage}
\usepackage{dcolumn}
\usepackage{times}
\usepackage[flushleft]{threeparttable}
\usepackage{tabularx}
\usepackage{booktabs}
\usepackage[12hr]{datetime}
\usepackage{longtable}
\usepackage[DIV=16]{typearea}
\usepackage{scrextend,booktabs}
\usepackage{tabulary}
\usepackage{array}
\usepackage{multirow}
\usepackage[font=small]{caption}


\begin{document}
	\begin{table}[htbp]
		\footnotesize
		\def\arraystretch{0.9}
		\centering
		\caption{The estimated test scores for Vietnam and Dev7 countries}
		\begin{tabular}{lcccccc}
    \toprule
    \midrule
          & \multicolumn{2}{c}{Math} & \multicolumn{2}{c}{Reading} & \multicolumn{2}{c}{Science} \\ \cline{2-7} \\
          & Mean (s.e.) & N     & Mean (s.e.) & N     & Mean (s.e.) & N \\
          \hline \\
    Vietnam & 511.34 & 4959  & 508.22 & 4959  & 528.42 & 4959 \\
          & (4.84) &       & (4.40) &       & (4.31) &  \\
    Developing 7 & 383.29 & 43524 & 405.06 & 43524 & 393.86 & 43524 \\
          & (2.50) &       & (2.46) &       & (2.25) &  \\
         \hline \\
         \multicolumn{7}{l}{Notes: The entries are means and standard errors of PISA data on the full sample,} \\
        \multicolumn{7}{l}{applying student weight. Student weights are provided in the data set. Standard errors}\\
         \multicolumn{7}{l}{in parentheses. The means comparison is coded with the intsvy package.} \\
      
    \end{tabular}%
  \label{tab:addlabel}%
\end{table}%
\end{document}
