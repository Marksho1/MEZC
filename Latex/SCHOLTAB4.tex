\documentclass[10pt]{article}
\usepackage{array}
\usepackage{longtable}
\usepackage{fullpage}
\usepackage{dcolumn}
\usepackage{times}
\usepackage[flushleft]{threeparttable}
\usepackage{tabularx}
\usepackage{booktabs}
\usepackage[12hr]{datetime}
\usepackage{longtable}
\usepackage[DIV=16]{typearea}
\usepackage{scrextend,booktabs}
\usepackage{tabulary}
\usepackage{array}
\usepackage{multirow}
\usepackage[font=small]{caption}

\begin{document}
	
\begin{table}[H]
	\footnotesize
	\def\arraystretch{0.9}
	\centering
	\caption{Summary statistics - school climate}
\begin{tabulary}{1.0\textwidth}{L L C C C C}
	\hline\hline \\
	\multicolumn{2}{c}{}
	& \multicolumn{2}{c}{Dev7 countries}
	& \multicolumn{2}{c}{Vietnam}	\\
	\hline & & & & & & 
	Variable & Description & MS & Valid N &  MS & Valid N \\
	\hline \\

STUDCLIM & Student-related aspects & 0.0485 & 40973 & 0.0418 & 4874 \\ 
& of school climate & (1.1642) &  & (0.6849) &  \\ 
TEACCLIM & Teacher-related aspects & -0.1997 & 40973 & -0.0873 & 4874 \\ 
& of school climate & (1.1474) &  & (0.7125) &  \\ 
TCMORALE & Teacher morale & 0.0376 & 41336 & -0.2941 & 4882 \\ 
& and enthusiasm & (1.0541) &  & (0.8579) &  \\ 
				
\hline \\
\multicolumn{6}{l}{Notes: The variables relate to the questionnaires administered to principals (schools). For a more}\\    
\multicolumn{6}{l}{detailed description of variables, please see Table xx. Items marked with \textit{(r)} are taken from}\\
\multicolumn{6}{l}{the rotated student questionnaire. The variable means of Dev7 and Vietnam are statistically}\\
\multicolumn{6}{l}{different at the 5\% significance level, except STUDCLIM.}\\

\end{tabulary}
\end{table}
	
	
\end{document}

