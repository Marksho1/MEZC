\documentclass[10pt]{article}
\usepackage{array}
\usepackage{longtable}
\usepackage{fullpage}
\usepackage{dcolumn}
\usepackage{times}
\usepackage[flushleft]{threeparttable}
\usepackage{tabularx}
\usepackage{booktabs}
\usepackage[12hr]{datetime}
\usepackage{longtable}
\usepackage[DIV=16]{typearea}
\usepackage{scrextend,booktabs}
\usepackage{tabulary}
\usepackage{array}
\usepackage{multirow}
\usepackage[font=small]{caption}

\begin{document}
	
\begin{table}[H]
	\footnotesize
	\def\arraystretch{0.9}
	\centering
	\caption{Summary statistics - student experience in mathematics}
\begin{tabulary}{1.0\textwidth}{L L C C C C}
	\hline\hline \\
	\multicolumn{2}{c}{}
	& \multicolumn{2}{c}{Dev7 countries}
	& \multicolumn{2}{c}{Vietnam}	\\
	\hline & & & & & & 
	Variable & Description & MS & Valid N &  MS & Valid N \\
	\hline \\
			 
 EXAPPLM \textit{(r)} & Experience with & 0.1111 & 26133 & -0.2418 & 3243 \\ 
 & applied math tasks & (1.06) &  & (0.7624) &  \\ [0.3em]
 EXPUREM \textit{(r)} & Experience with pure & -0.1384 & 25973 & 0.1587 & 3244 \\ 
 & math tasks & (0.9809) &  & (0.8076) &  \\ [0.3em]
 FAMCONC \textit{(r)} & Familiarity with & -0.5441 & 25832 & 0.4297 & 3231 \\ 
 & math concepts & (0.8768) &  & (0.9057) &  \\ [0.3em]

\hline \\
\multicolumn{6}{l}{Notes: The variables relate to the questionnaires administered to students in the rotated}\\   
\multicolumn{6}{l}{booklet. For a more detailed description of variables, please see Table xx. Items marked}\\    
\multicolumn{6}{l}{with \textit{(r)} are taken from the rotated student questionnaire. The variable means of}\\
\multicolumn{6}{l}{Dev7 and Vietnam are statistically different at the 5\% significance level.}\\

\end{tabulary}
\end{table}
	
	
\end{document}

