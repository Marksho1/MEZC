\documentclass[10pt]{article}
\usepackage{array}
\usepackage{longtable}
\usepackage{fullpage}
\usepackage{dcolumn}
\usepackage{times}
\usepackage[flushleft]{threeparttable}
\usepackage{tabularx}
\usepackage{booktabs}
\usepackage[12hr]{datetime}
\usepackage{longtable}
\usepackage[DIV=16]{typearea}
\usepackage{scrextend,booktabs}
\usepackage{tabulary}
\usepackage{array}
\usepackage{multirow}
\usepackage[font=small]{caption}

\begin{document}

\begin{table}[htbp]
\footnotesize
	\def\arraystretch{0.9}
  \centering
  \caption{The estimated test score gap between Vietnam and Dev7 countries}
    \begin{tabular}{lccc}
     \toprule
     \midrule
     & \multicolumn{1}{c}{Math} & \multicolumn{1}{c}{Reading} & \multicolumn{1}{c}{Science} \\
     \hline \\
     Intercept & 328.29 & 403.06 & 393.86 \\
     & (2.5) & (2.46) & (2.25) \\
     Vietnam & 128.05 & 105.16 & 134.56 \\
     & (27.21) & (5.03) & (4.91) \\ 
     R-squared & 27.21 & 19.61 & 30.75 \\
     Number of observations &  & 48,483 &\\
     \hline \\
     \multicolumn{4}{l}{Notes: The dependent variable is Mathematics, Reading} \\
     \multicolumn{4}{l}{or Science score (all 5 plausible values). Regressions are} \\
     \multicolumn{4}{l}{coded with the intsvy package. The unit of observation is a} \\
     \multicolumn{4}{l}{student. Student weights are provided in the data set} \\
     \multicolumn{4}{l}{(W\_FSTUWT). Standard errors in parentheses.} \\
      
        \end{tabular}%
        \label{tab:addlabel}%
    \end{table}%
    
\end{document}